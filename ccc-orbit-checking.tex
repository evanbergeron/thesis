\documentclass{article}
% \usepackage[margin=1in]{geometry}
\usepackage{amsmath, amssymb}
\usepackage{parskip}
\usepackage{enumerate}
\usepackage[T1]{fontenc}

\title{Notes on Transducer Orbit Checking}
\author{Evan Bergeron}
\begin{document}
\maketitle
\subsection*{A Necassary Condition}
Let $f$ be the function produced by $A^3_2$. Let $y \in f^*(x)$. Let $c$ be a sequence where $c_i$ is the number of times $x_i$ is flipped on the way to $y$. Then we claim that
\begin{align*}
c_i &= \lfloor (c_{i-3} / 2) \rfloor + (c_{i-3} \mod{2}) \cdot (1 - x_{i-3})\\
    &+\lfloor (c_{i-2} / 2) \rfloor + (c_{i-2} \mod{2}) \cdot x_{i-2}
\end{align*}
where $c_0$ is the index of $y$ in $f^*(x)$, $c_1$ is 0, and $c_2$ is $\lfloor c_0 / 2 \rfloor$.

$x_0$ is flipped upon every invocation of $f$. $x_1$ is never flipped. $x_2$ is flipped roughly every other time $x_0$ is flipped. That is, it's flipped every time $x_0$ changes from a 1 to a 0. Without loss, we may assume $x_0 = 0$. Thus, $c_2 = \lfloor c_0 / 2 \rfloor$ (it's not flipped the first time, but then is flipped every other time afterward).

Consider some application of $f$. $c_i$ is flipped iff $c_{i-2}$ is flipped from a 1 to a 0 or $c_{i-3}$ is flipped from a 0 to a 1. If $c_{i-2}$ and $c_{i-3}$ are even, then this is precisely every other flip. If either $c_{i-2}$ or $c_{i-3}$ is odd, then $c_i$ is dependent on both $c_{i-2}$ and $c_{i-3}$ as well as the initial conditions in $x$. We add one depending on whether or not the first flip of $c_{i-2}$, $c_{i-3}$ causes $c_i$ to flip as well.

\subsection*{A Sufficient Condition}
Let $p$ be a sequence where $p_i = c_i \mod{2}$ for all $i$. Then if $x \oplus y$ looks like some $p$, then $y \in f^*(x)$. That is, if you fix your initial conditions and the above recurrence holds through $x \oplus y$, then $y \in f^*(x)$.

That being said, we necassarily don't know $c_0$.

\subsection*{$A^3_2$ Orbit Checking is in NP}
Our verifier takes in two strings $x$, $y$, and an index $i$. This index is the position of $y$ in $x$'s orbit. WLOG, suppose $x_0 = 1$. We first set $c_0 = i$, $c_1 = 0$, and $c_2 = \lfloor c_0 / 2 \rfloor$. We then calculate $c_i$ and check that $c_i \mod 2 = x_i \oplus y_i$ for all $i$.

The certificate is poly length with respect to $x$ and $y$, as the orbit of $y$ has length at most $2^n$ (so the length of an index is at most $n$).

\subsection*{A Rephrasing of the Problem}
In orbit iff there is a $t$ such that $y = f^t(x)$. Suffices to find this $t$.

\subsection*{Derivatives}
Notation incoming.

\subsection*{$A^3_2$ Orbit Checking is in P}
The algo in question.

\subsection*{The $A^3_2$ Orbit Relation is Rational}
This is hard - need to write that vector reduction thing.

\subsection*{A New Class of Transducers - 1-Toggle-1-Split}

\subsection*{1-Tog-1-Split Orbit Checking in P?}

\subsection*{1-Tog-1-Split Orbit Relation Rational?}

\subsection*{1-Tog Orbit Relation Rational?}

TODO: Automate making the orbit automaton?

\end{document}
