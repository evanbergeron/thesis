\documentclass{article}
% \usepackage[margin=1in]{geometry}
\usepackage{amsmath, amssymb, amsthm}
% \usepackage{parskip}
\usepackage{tikz}
\usepackage{multicol}
\usepackage{enumerate}
\usepackage{xltxtra}
% \usepackage{charter}
% \usepackage[T1]{fontenc}

\setmainfont{Equity Text B}
\renewcommand{\baselinestretch}{1.2}

%%%%%%%%% Useful shorthand %%%%%%%%%%%%
\newcommand{\0}{\underline{0}}
\newcommand{\1}{\underline{1}}
\newcommand{\2}{\underline{2}}
\newcommand{\N}{\mathbb{N}}
\newcommand{\Z}{\mathbb{Z}}
\renewcommand{\S}{\mathcal{S}}
%%%%%%%%%%%%%%%%%%%%%%%%%%%%%%%%%%%%%%%

\title{Notes osdfn Transducer Orbit Checking}
\author{Evan Bergeron}
\begin{document}
\maketitle

\subsection*{A Simple Invertible Transducer}
This is $A^3_2$.
\begin{center}
\begin{tikzpicture}[scale=0.1]
\tikzstyle{every node}+=[inner sep=0pt]
\draw [black] (39.5,-18.5) circle (3);
\draw (39.5,-18.5) node {$0$};
\draw [black] (50.8,-38.4) circle (3);
\draw (50.8,-38.4) node {$2$};
\draw [black] (27.5,-38.4) circle (3);
\draw (27.5,-38.4) node {$1$};
\draw [black] (40.322,-21.38) arc (10.02185:-72.20308:14.565);
\fill [black] (30.43,-37.78) -- (31.34,-38.01) -- (31.04,-37.06);
\draw (39.09,-32.7) node [right] {$1/0$};
\draw [black] (42.467,-18.9) arc (75.86616:-16.68694:13.324);
\fill [black] (51.98,-35.65) -- (52.68,-35.02) -- (51.73,-34.74);
\draw (51.46,-24.02) node [right] {$0/1$};
\draw [black] (48.761,-40.593) arc (-48.81225:-131.18775:14.594);
\fill [black] (29.54,-40.59) -- (29.81,-41.5) -- (30.47,-40.74);
\draw (39.15,-44.7) node [below] {$a/a$};
\draw [black] (26.496,-35.579) arc (-166.604:-255.57722:13.874);
\fill [black] (36.54,-18.93) -- (35.64,-18.64) -- (35.89,-19.61);
\draw (27.47,-23.93) node [left] {$a/a$};
\end{tikzpicture}
\end{center}
Let $\underline{i}$ be the function induced by applying $A^3_2$ to some string with state $i$ as the start state. $\0, \1, \2$ generate a semigroup under composition. For example, $\0^2$ is the result of applying $\0$ twice. We call such functions \textit{transductions}.

If $x$ is some string, call $\0^* = \{ \0^t(x) \mid t \in \N \}$ the \textit{orbit} of $x$ under $\0$.

We prove that checking if $y \in \0^*(x)$ is in P. We then investigate orbit checking for arbitrary $f$ in the semigroup.

\subsection*{A Necassary Condition}
Let $f = \0$. Let $y \in f^*(x)$. Let $c$ be a sequence where $c_i$ is the number of times $x_i$ is flipped on the way to $y$. Then we claim that
\begin{align*}
c_i &= \lfloor (c_{i-3} / 2) \rfloor + (c_{i-3} \mod{2}) \cdot (1 - x_{i-3})\\
    &+\lfloor (c_{i-2} / 2) \rfloor + (c_{i-2} \mod{2}) \cdot x_{i-2}
\end{align*}
where $c_0$ is the index of $y$ in $f^*(x)$, $c_1$ is 0, and $c_2$ is $\lfloor c_0 / 2 \rfloor$.

$x_0$ is flipped upon every invocation of $f$. $x_1$ is never flipped. $x_2$ is flipped roughly every other time $x_0$ is flipped. That is, it's flipped every time $x_0$ changes from a 1 to a 0. Without loss, we may assume $x_0 = 0$. Thus, $c_2 = \lfloor c_0 / 2 \rfloor$ (it's not flipped the first time, but then is flipped every other time afterward).

Consider some application of $f$. $c_i$ is flipped iff $c_{i-2}$ is flipped from a 1 to a 0 or $c_{i-3}$ is flipped from a 0 to a 1. If $c_{i-2}$ and $c_{i-3}$ are even, then this is precisely every other flip. If either $c_{i-2}$ or $c_{i-3}$ is odd, then $c_i$ is dependent on both $c_{i-2}$ and $c_{i-3}$ as well as the initial conditions in $x$. We add one depending on whether or not the first flip of $c_{i-2}$, $c_{i-3}$ causes $c_i$ to flip as well.

\subsection*{A Sufficient Condition}
Let $p$ be a sequence where $p_i = c_i \mod{2}$ for all $i$. Then if $x \oplus y$ looks like some $p$, then $y \in f^*(x)$. That is, if you fix your initial conditions and the above recurrence holds through $x \oplus y$, then $y \in f^*(x)$.

That being said, we necassarily don't know $c_0$.

\subsection*{$\0$ Orbit Checking is in NP}
Our verifier takes in two strings $x$, $y$, and an index $i$. This index is the position of $y$ in $x$'s orbit. WLOG, suppose $x_0 = 1$. We first set $c_0 = i$, $c_1 = 0$, and $c_2 = \lfloor c_0 / 2 \rfloor$. We then calculate $c_i$ and check that $c_i \mod 2 = x_i \oplus y_i$ for all $i$.

The certificate is poly length with respect to $x$ and $y$, as the orbit of $y$ has length at most $2^n$ (so the length of an index is at most $n$).


% \subsection*{A Rephrasing of the Problem}
% In orbit iff there is a $t$ such that $y = f^t(x)$. Suffices to find this $t$.

\subsection*{Semigroups Revisited}
Recall that the states of $A^3_2$ form a semigroup $\S(A^3_2)$. This semigroup is commutative - it doesn't matter what order we apply the transductions in. This follows by commutativity of addition modulo 2 (suffices to consider each bit one at a time). This means that every element in $\S(A^3_2)$ looks like $\0^i \1^j\2^k$ for $i, j, k \in \N$.

Further, we have an identity element in $\S(A^3_2)$: $$\0^2\1^2\2 = I$$
\begin{proof}
% TODO - not sure how this one works. It suffices to show that $c_i$ is even for all $i$.
Proof by induction - it suffices to show that $c_i$ is even for all $i$. In fact, $c_i$ will be 2.

$c_0$ is 2, each of the applications of $\0$ flip $x_0$ once. The applications of $\1$ and $\2$ skip it. $c_1$ is also 2 as each application of $\1$ flips $x_1$ once and $\0$ and $\2$ skip it. $c_2 = 2$ - $\2$ flips it once, $\1^2$ skips it, and exactly one of the two applications of $\0$ flip it.

Then $c_{i-2} = c_{i-3} = 2$ by induction, so
\begin{align*}
c_i &= \lfloor (c_{i-3} / 2) \rfloor + \lfloor (c_{i-2} / 2) \rfloor = 1 + 1 = 2
\end{align*}
\end{proof}
This means that $\S(A^3_2)$ is already a group ($\0^{-1} = \0\1^2\2$ and so on). Further, since $\2$ is expressible in terms of $\0$ and $\1$, we have that $\S(A^3_2) = \{ \0^i \1^j \mid i, j \in \Z \}$, giving us a concise data structure to represent elements of $\S(A^3_2)$.

% TODO get a nice distinction between the two representations

\subsection*{Derivatives}
We introduce the notion of so-called derivatives. They're named as such because they resemble normal calculus derivatives in many ways, though that's currently beyond my knowledge.

If $f_u$ is the transduction corresponding to some state state $u$ of a transducer, then $\partial_b(f_u) = f_v$ if state $u$ transitions to state $v$ on input $b$. Each transducer has a corresponding derivative table, which is effectively an adjacency list representation of the transducer.

For example, $A^3_2$ has the following representation:

\begin{multicols}{2}
\begin{itemize}
\item $\partial_0(\0) = \2$
\item $\partial_1(\0) = \1$
\item $\partial_b(\2) = \1$ for any $b$
\item $\partial_b(\1) = \0$ for any $b$
\end{itemize}

\begin{itemize}
\item $\partial_0(\0^i) = \1^{\lfloor i / 2 \rfloor}\2^{\lceil i / 2 \rceil}$
\item $\partial_1(\0^i) = \1^{\lceil i / 2 \rceil}\2^{\lfloor i / 2 \rfloor}$
\item $\partial_b(\1^i) = \0^i$ for any $b$
\item $\partial_b(\2^i) = \1^i$ for any $b$
\item $\partial_0(\0^i\1^j\2^k) = \0^j\1^{\lfloor i / 2 \rfloor+k}\2^{\lceil i / 2 \rceil}$
\item $\partial_1(\0^i\1^j\2^k) = \0^j\1^{\lceil i / 2 \rceil+k}\2^{\lfloor i / 2 \rfloor}$
\end{itemize}
\end{multicols}
% TODO why does this work this way?
We can extend this notion of differentiation to differentiating with respect to arbitrary strings, rather than single bits (simply iterate the differentiation). Then, for $x, y \in \textbf{1}^*$, $f \in \S$
$$f(xy) = f(x) \partial_xf(y)$$

\subsection*{$A^3_2$ Orbit Checking is in NP}

Call $f \in \S(A^3_2)$ \textit{even} if $f$ leaves the first bit of its input unchanged and \textit{odd} otherwise.

It then suffices, given $i$ as the index into the orbit as a certificate, $f \in \S(A^3_2)$ and strings $x, y$, to simply iterate through $x$, differentiating $f$ as we go and asserting that $y_i = \partial_zf(x_i)$, where $z$ is the prefix of $x$ so far.

Parity checking $f$ can be done in polynomial time - simply check the parity of $\0$ in $f$. Additionally, differentiation can be performed in polynomial time, as it's a couple of additions and a couple bit shifts. Since the value of $i$ is at most $2^n$ where $n$ is the input length, the length of $i$ is polynomial with respect to $x$ and $y$.

\subsection*{$A^3_2$ Orbit Checking is in P}

\subsection*{The $A^3_2$ Orbit Relation is Rational}
This is hard - need to write that vector reduction thing.

\subsection*{A New Class of Transducers - 1-Toggle-1-Split}

\subsection*{1-Tog-1-Split Orbit Checking in P?}

\subsection*{1-Tog-1-Split Orbit Relation Rational?}

\subsection*{1-Tog Orbit Relation Rational?}

TODO: Automate making the orbit automaton?

\end{document}
