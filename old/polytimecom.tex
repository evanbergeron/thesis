\documentclass[11pt]{article}
% \usepackage[left=1in, right=2.5in, bottom=1.25in, top=1in]{geometry}
\usepackage{amsmath, amssymb, amsthm}
\usepackage{tikz}
\usepackage{multicol}
\usepackage{titlesec}
\usepackage{fancyhdr}
\usepackage{enumerate}
\usepackage{xltxtra}

% Font stuff
\newcommand*{\concourse}{\fontspec[]{Concourse T4}\selectfont}
\newcommand*{\caps}{\fontspec[]{Equity Caps B}\selectfont}
\renewcommand*{\textsc}{\caps}

\pagestyle{fancy}
\fancyhead{}
\renewcommand{\headrulewidth}{0pt} % no line
\fancyhead[LO]{\leftmark}
\fancyhead[RE]{\rightmark}
\fancyhead[LE, RO]{\thepage}
\fancyfoot{}
% \lfoot{\thepage}

% Theorem stuff
\newtheoremstyle{pleasant} % Name
  {\topsep}                % Space above
  {\topsep}                % Space below
  {\itshape}               % Body font
  {}                       % Indent amount
  {\small\caps}               % Theorem head font
  {.}                      % Punctuation after theorem head
  {.5em}                   % Space after theorem head
  {}                       % Theorem head spec (empty means normal)

\theoremstyle{pleasant}
\newtheorem{proposition}{Proposition}
\newtheorem{theorem}{Theorem}
\newtheorem{corollary}{Corollary}
\newtheorem{lemma}{Lemma}
\newtheorem{definition}{Definition}
\newtheorem{question}{Question}
\newtheorem{conjecture}{Conjecture}

% Make titles prettier
\titleformat*{\subsection}{\large\caps}
\titleformat*{\subsubsection}{\normalsize\caps}
% left, before, after
\titlespacing*{\subsection}{0pt}{1.75em plus 0.5em minus 0.2em}{1em}

% More font stuff
\setmainfont{Equity Text B}
\renewcommand{\baselinestretch}{1.2}

%%%%%%%%% Useful shorthand %%%%%%%%%%%%
\newcommand{\0}{\underline{0}}
\newcommand{\1}{\underline{1}}
\newcommand{\2}{\underline{2}}
\newcommand{\N}{\mathbb{N}}
\newcommand{\Z}{\mathbb{Z}}
\renewcommand{\S}{\mathcal{S}}
%%%%%%%%%%%%%%%%%%%%%%%%%%%%%%%%%%%%%%%

\title{\concourse{A polynomial time algorithm for determining if a transducer's associated semigroup is Abelian}}
\author{\concourse{Tsutomu Okano \& Evan Bergeron}}
\date{\concourse{\today}}
\begin{document}
\maketitle

% \chapter*{\normalfont\concourse{A polynomial time algorithm for determining if a transducer's associated semigroup is Abelian}}
% \chapter*{\normalfont\concourse{Determining commutativity of invertible transducers}}

% \vspace{-1em}
% {\Large\concourse{Tsutomu Okano \& Evan Bergeron}}

% \subsection*{Definitions}
% A semigroup $S$ is called an \textit{automaton semigroup} iff there is some transducer $\mathcal{A}$ such that $S \cong S(\mathcal{A})$.

% An transducer is said to be \textit{invertible} iff the output function of each state is a permutation of the input alphabet. Over the binary alphabet, a state whose output function is the identity is called a \textit{copy state}. Otherwise, it is called a \textit{toggle state}.

% Fix some state $s$. The state to which it transitions when reading a 0 is denoted $\partial_0 t$. Similarly, we have $\partial_1 t$.

% The states of a transducer $A$ induce functions of type $\Sigma^* \rightarrow \Sigma^*$. These functions form a semigroup $S(A)$ when closed under composition.

% The main result of this draft is a polynomial time algorithm for determining whether $S(A)$ is commutative for some invertible transducer $A$.

We have the following characterization from Okano: $S(A)$ is Abelian iff there exists a unique $\Theta_A \in S(A)$ such that $\partial_0 \underline{t} = \partial_1 \underline{t} \Theta_A$ for all toggle states $t$. So it suffices to verify that each toggle state obeys this.

Since these transducers are synchronous (output exactly one character per transition), we can build a DFA that recognizes the input-output relation of some fixed starting state. Take some $s$ this start state. Then build a DFA over the alphabet $\Sigma \times \Sigma$ with transitions simulating the behavior of the transducer starting at $s$. This DFA is called the \textit{acceptor of $A$ at $s$}.

An automaton semigroup $S(A)$ has a presentation with generators corresponding to the states of $A$. Thinking of elements of $S(A)$ as words over this generator alphabet, we see that the state set of $A$ is precisely the words of length 1. We can build an automaton whose state set is the set of all length 2 words as follows:

If $A = (Q, \Sigma, \delta)$, the product automaton $A \times A$ has state set $Q\times Q$ with transition function $\partial_a (s_1, s_2) = (\partial_a s_1, \partial_{a s_1} s_2)$. We can see by induction that each state $(s_1, s_2)$ corresponds to the word $s_1 s_2 \in S(A)$.

Remember that $(\partial_1 \underline{t}) = (\partial_0 \underline{t})\Theta_A$. So then we have $(\partial_1 \underline{t})(\partial_0 \underline{t})^{-1} = \Theta_A$. (This is straight from Okano's thesis, I think this expression is better written as $(\partial_0 \underline{t})^{-1}(\partial_1 \underline{t})= \Theta_A$ - it seems more consistent with function-application-from-the-right). 

Anyway, $\Theta_A = (\partial_1 \underline{t})(\partial_0 \underline{t})^{-1} = (\partial_1 \underline{t})(\partial_1 \underline{t^{-1}})$, where $t^{-1}$ lies in the inverse automaton for $A$. So then build for each toggle state the following: $(A\times A^{-1})(\partial_1 \underline{t}, \partial_1 \underline{t^{-1}})$. Note that the language of this automaton is $\{ x : y \mid y = x(\partial_1 \underline{t})(\partial_0 \underline{t})^{-1}\}$. It suffices to then verify that all of these constructed DFAs are equivalent, which can be done via Hopcroft's minimization algorithm.

So to summarize algorithmically: take the input automaton $A$. Build it's inverse automaton $A^{-1}$. Construct the product automaton $A\times A^{-1}$. Then for each toggle state $t_i$ of $A$, take the state $s_i = (\partial_1 t_i, \partial_1 t_i^{-1})$ in $A\times A^{-1}$ and construct the DFA $(A\times A^{-1})(s_i)$. Verify all the constructed DFAs are equivalent.

% Interestingly, note that this approach leads to (new?) proof of the following claim:

This product automaton construction also gives us that the word problem for automaton semigroups is decidable.

% \textit{Claim:} All automaton semigroups have a decidable word problem. \textit{Proof:} Take the two words $u$ and $v$ over the state set alphabet. Construct product automata $A^{|u|}$, $A^{|v|}$. Construct the acceptors $A^{|u|}(u)$ and $A^{|v|}(v)$. Then use Hopcroft minimization to verify that the two DFAs are equivalent.

\end{document}
