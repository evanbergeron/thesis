\documentclass[11pt]{article}
% \usepackage[left=1in, right=2.5in, bottom=1.25in, top=1in]{geometry}
\usepackage{amsmath, amssymb, amsthm}
\usepackage{tikz}
\usepackage{multicol}
\usepackage{titlesec}
\usepackage{fancyhdr}
\usepackage{enumerate}
\usepackage{xltxtra}

% Font stuff
\newcommand*{\concourse}{\fontspec[]{Concourse T4}\selectfont}
\newcommand*{\caps}{\fontspec[]{Equity Caps B}\selectfont}
\renewcommand*{\textsc}{\caps}
\newcommand{\fstar}{f^*}
\newcommand{\kstar}{^*}

% \pagestyle{fancy}
% \fancyhead{}
% \renewcommand{\headrulewidth}{0pt} % no line
% \fancyhead[LO]{\leftmark}
% \fancyhead[RE]{\rightmark}
% \fancyhead[LE, RO]{\thepage}
% \fancyfoot{}
% % \lfoot{\thepage}

% Theorem stuff
\newtheoremstyle{pleasant} % Name
  {\topsep}                % Space above
  {\topsep}                % Space below
  {\itshape}               % Body font
  {}                       % Indent amount
  {\small\caps}               % Theorem head font
  {.}                      % Punctuation after theorem head
  {.5em}                   % Space after theorem head
  {}                       % Theorem head spec (empty means normal)

\theoremstyle{pleasant}
\newtheorem{proposition}{Proposition}
\newtheorem{theorem}{Theorem}
\newtheorem{corollary}{Corollary}
\newtheorem{lemma}{Lemma}
\newtheorem{definition}{Definition}
\newtheorem{question}{Question}
\newtheorem{conjecture}{Conjecture}

% Make titles prettier
\titleformat*{\section}{\Large\caps}
\titleformat*{\subsection}{\large\caps}
\titleformat*{\subsubsection}{\normalsize\caps}
% left, before, after
\titlespacing*{\subsection}{0pt}{1.75em plus 0.5em minus 0.2em}{1em}

% More font stuff
\setmainfont{Equity Text B}
\renewcommand{\baselinestretch}{1.2}

%%%%%%%%% Useful shorthand %%%%%%%%%%%%
\newcommand{\0}{\underline{0}}
\newcommand{\1}{\underline{1}}
\newcommand{\2}{\textbf{2}}
\newcommand{\N}{\mathbb{N}}
\newcommand{\Z}{\mathbb{Z}}
\renewcommand{\S}{\mathcal{S}}
\renewcommand{\G}{\mathcal{G}}
%%%%%%%%%%%%%%%%%%%%%%%%%%%%%%%%%%%%%%%

\title{\concourse{SCS Senior Thesis Prospectus}}
\author{\concourse{Evan Bergeron}\\
\concourse{Advisor: Klaus Sutner}}
\date{\concourse{\today}}
\begin{document}
\maketitle

\begin{abstract}
We intend to study iterated transductions defined by a class of invertible transducers over the binary alphabet. Previous work has investigated orbit checking algorithms and the orbit rationality problem for a subclass of automata associated with Abelian free groups and monoids of finite rank.

We intend to answer these same questions for increasingly large subclasses of invertible transducers.
\end{abstract}
\section{Problem Description / Significance}

An invertible transducer is a Mealy automaton where all transitions are of the form $p^{a/\pi_p(a)}\rightarrow q$, where $\pi_p$ is a permutation of the alphabet depending on the source state $p$. We only consider $\textbf{2} = \{ 0, 1 \}$ as input and output alphabet. Choosing an arbitrary start state $p$, we obtain a transduction from $\2^*$ to $\2^*$. These transductions form a semigroup $\S(A)$. Including inverses, we obtain groups $\G(A)$. These groups are called automata groups or self-similar groups, studied in great detail in group theory and symbolic dynamics. For instance, Grigorchuk's group of intermediate growth demonstrates the descriptive power of invertible transducers.
\section{Proposed Research Plan}

A good deal of background reading has already been completed. I've read 2 papers by Klaus, as well as Okano's thesis from last year. Further reader still needs to be done, in particular \cite{NekrashevychSidki04:automorphisms}.

I'll be working alongside Klaus to produce original results. We expect to answer the orbit rationality question for subclasses of inverse transducers with associated Abelian semigroups and multiple toggle states; a class of automata not previously examined in this way.

%We will also be investigating transducers with single toggle states, but non-Abelian semigroups.

We also hope to characterize the behaviour of several of the non-Abelian cases as well; though we expect these orbit relations will fail to be rational.

\subsection{Timeline}
\begin{itemize}
\item September: reading, absorbing known techniques.

\item October: study special cases with a view towards handling simple non-commutative automata.

\item November: extend orbit characterization in commutative cases.

\item December-January: develop and implement decision algorithm for orbit rationality for as large a class of automata as feasible.

\item February - March: further develop connections between group theoretic and model theoretic properties of invertible automata.

\item April: Writeup.
\end{itemize}

\nocite{*}
\bibliographystyle{unsrt}
\bibliography{refs}

\end{document}
