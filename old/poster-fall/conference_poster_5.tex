\documentclass[a0,16pt]{a0poster}

\usepackage{multicol}
\columnsep=100pt
% \usepackage{xltxtra} % xelatex goodies, plus font selection
\usepackage{tabularx}
% \usepackage{parskip}
\usepackage{titlesec}

% Specify colors by their 'svgnames',
% for a full list of all colors available see here:
% http://www.latextemplates.com/svgnames-colors
\usepackage[svgnames]{xcolor}

\renewcommand{\baselinestretch}{1.2}
\usepackage{palatino} % Uncomment to use the Palatino font
% \setmainfont{Equity Text B}

\newcommand*{\concourse}{}
\newcommand*{\concoursecaps}{\textsc}
\newcommand*{\caps}{\textsc}

\usepackage{graphicx} % Required for including images
\usepackage{tikz} % Required for including images
\graphicspath{{figures/}} % Location of the graphics files
\usepackage{booktabs} % Top and bottom rules for table

% Required for specifying captions to tables and figures
\usepackage[font=small,labelfont=bf]{caption}
\usepackage{amsfonts, amsmath, amsthm, amssymb} % For math fonts, symbols and environments
\usepackage{wrapfig} % Allows wrapping text around tables and figures

% Theorem stuff
\newtheoremstyle{pleasant} % Name
  {\topsep}                % Space above
  {\topsep}                % Space below
  {\itshape}               % Body font
  {}                       % Indent amount
  {\concourse}               % Theorem head font
  {.}                      % Punctuation after theorem head
  {.5em}                   % Space after theorem head
  {}                       % Theorem head spec (empty means normal)
\theoremstyle{pleasant}
\newtheorem{proposition}{Proposition}
\newtheorem{theorem}{Theorem}
\newtheorem{corollary}{Corollary}
\newtheorem{lemma}{Lemma}
\newtheorem{definition}{Definition}
\newtheorem{question}{Question}
\newtheorem{conjecture}{Conjecture}

\newenvironment{proofsketch}{\paragraph{\Large \normalfont \textit{Proof Sketch:}}}{\hfill$\square$}

% Make titles prettier
\titleformat*{\section}{\huge\concourse}
\titleformat*{\subsection}{\Large\concourse}
\titleformat*{\subsubsection}{\concourse}

% \fontsize{80}{50}\selectfont

% left, before, after
\titlespacing*{\section}{0pt}{1.2em plus 0.5em minus 0.2em}{1em}
\titlespacing*{\subsection}{0pt}{1.75em plus 0.5em minus 0.2em}{1em}

%%%%%%%%% Useful shorthand %%%%%%%%%%%%
\newcommand{\0}{\underline{0}}
\newcommand{\1}{\underline{1}}
\newcommand{\2}{\underline{2}}
\newcommand{\N}{\mathbb{N}}
\newcommand{\Z}{\mathbb{Z}}
\renewcommand{\S}{\mathcal{S}}
%%%%%%%%%%%%%%%%%%%%%%%%%%%%%%%%%%%%%%%

\begin{document}

%----------------------------------------------------------------------------------------
%	POSTER HEADER
%----------------------------------------------------------------------------------------

% The header is divided into three boxes:
% The first is 55% wide and houses the title, subtitle, names and university/organization
% The second is 25% wide and houses contact information
% The third is 19% wide and houses a logo for your university/organization or a photo of you
% The widths of these boxes can be easily edited to accommodate your content as you see fit

\begin{minipage}[b]{0.55\linewidth}
\veryHuge \color{NavyBlue} \concourse{Decision Problems for Automaton Semigroups} \color{Black}\\ % Title
% \Huge\textit{An Exploration of Complexity}\\[1cm] % Subtitle
\huge \concourse{Evan Bergeron \& Klaus Sutner}\\ % Author(s)
\huge Carnegie Mellon Computer Science Department\\ % University/organization
\end{minipage}


% \vspace{1cm} % A bit of extra whitespace between the header and poster content

%--------------------------------------------------------------------------------------

\begin{multicols}{3} % This is how many columns your poster will be broken into, a poster with many figures may benefit from less columns whereas a text-heavy poster benefits from more

%--------------------------------------------------------------------------------------
%	ABSTRACT
%--------------------------------------------------------------------------------------

% \color{Navy} % Navy color for the abstract

\begin{abstract}
\large
The word problem is a classic group-theoretic decision problem. It's known to be undecidable in surprisingly small subclasses of groups. We consider a class of semigroups generated by finite automata for which this problem is decidable. We consider several related decision problems for this subclass of semigroups.
\end{abstract}

%----------------------------------------------------------------------------------------
%	INTRODUCTION
%----------------------------------------------------------------------------------------

%	RESULTS
%----------------------------------------------------------------------------------------

% \section*{Results}
% \subsection*{A Simple Invertible Transducer}
\section*{Automaton Semigroups}

\Large
Below is an example \textit{invertible transducer}. It's quite similar to a DFA, but instead of just reading in a string, it also outputs a string (and so induces a relation on strings).

\begin{center}
\begin{tikzpicture}[scale=0.3]
\tikzstyle{every node}+=[inner sep=0pt]
\draw [black] (39.5,-18.5) circle (3);
\draw (39.5,-18.5) node {$0$};
\draw [black] (50.8,-38.4) circle (3);
\draw (50.8,-38.4) node {$2$};
\draw [black] (27.5,-38.4) circle (3);
\draw (27.5,-38.4) node {$1$};
\draw [black] (40.322,-21.38) arc (10.02185:-72.20308:14.565);
\fill [black] (30.43,-37.78) -- (31.34,-38.01) -- (31.04,-37.06);
\draw (39.09,-32.7) node [right] {$1/0$};
\draw [black] (42.467,-18.9) arc (75.86616:-16.68694:13.324);
\fill [black] (51.98,-35.65) -- (52.68,-35.02) -- (51.73,-34.74);
\draw (51.46,-24.02) node [right] {$0/1$};
\draw [black] (48.761,-40.593) arc (-48.81225:-131.18775:14.594);
\fill [black] (29.54,-40.59) -- (29.81,-41.5) -- (30.47,-40.74);
\draw (39.15,-44.7) node [below] {$a/a$};
\draw [black] (26.496,-35.579) arc (-166.604:-255.57722:13.874);
\fill [black] (36.54,-18.93) -- (35.64,-18.64) -- (35.89,-19.61);
\draw (27.47,-23.93) node [left] {$a/a$};
\end{tikzpicture}
\end{center}
There's a natural correspondence between states in the machine and length-preserving functions from string to string. For instance, if $\underline{0}$ is the function induced by starting at state 0, we have $\underline{0}(0000) = 1001$.

These functions form a semigroup $\S(A)$ under composition (associative and closed).

More formally, let $\textbf{2}$ be the binary alphabet, $Q$ be the state set of the automaton $A$, $Q^+$ be all nonempty strings over $Q$, and $End(\textbf{2}^*)$ be the semigroup of endomorphisms on the tree $\textbf{2}^*$.

Then there's a natural homomorphism $\phi : Q^+ \rightarrow End(\textbf{2}^*)$. We denote the image of $\phi$ as $\Sigma(A)$.  

% For example, given the string 0000, we have
% $$0000 \rightarrow 1001 \rightarrow 0011 \rightarrow 1010\rightarrow 0000$$

% This can be modeled with the following finite state machine $A$:\\

% When we have two states $p$ and $q$ with the transition $p ^{a/b}\rightarrow q$, this means ``read character $a$ and output character $b$''. % TODO fix this arrow
% The algorithm previously described starts on state 0.

% Starting at the 1st state ignores the first character, then proceeds as normal. We can view these pebble flipping algorithm as a function from string to string. 

\begin{definition}
% Given some function $f \in \S(A)$ and some string $x$, the set of all strings reachable from $x$ by repeated application of $f$ is called the {\normalfont\concourse{orbit of $x$ under $f$}}.
A semigroup $S$ is called an \textbf{automaton semigroup} if there exists an automaton $A$ such that $S \simeq \Sigma(A)$.
\end{definition}

Many natural classes of semigroups arise as automaton semigroups. For instance, all free semigroups of rank $\geq 2$ arise as automaton semigroups. 
% Below is an automaton generating the commutative semigroup of rank 3.

% \begin{center}
% \begin{tikzpicture}[scale=0.4]
% \tikzstyle{every node}+=[inner sep=0pt]
% \draw [black] (41.9,-27.1) circle (3);
% \draw (41.9,-27.1) node {$0$};
% \draw [black] (55.1,-27.1) circle (3);
% \draw [black] (32.4,-35.3) circle (3);
% \draw (32.4,-35.3) node {$2$};
% \draw [black] (32.4,-17.6) circle (3);
% \draw (32.4,-17.6) node {$1$};
% \draw [black] (44.9,-27.1) -- (52.1,-27.1);
% \fill [black] (52.1,-27.1) -- (51.3,-26.6) -- (51.3,-27.6);
% \draw (48.5,-26.6) node [above] {$1/0$};
% \draw [black] (57.78,-25.777) arc (144:-144:2.25);
% \draw (62.35,-27.1) node [right] {$a/a$};
% \fill [black] (57.78,-28.42) -- (58.13,-29.3) -- (58.72,-28.49);
% \draw [black] (41.06,-29.967) arc (-25.51166:-72.88958:9.372);
% \fill [black] (35.36,-34.89) -- (36.27,-35.13) -- (35.98,-34.17);
% \draw (40.89,-33.52) node [below] {$0/1$};
% \draw [black] (35.363,-17.978) arc (73.34933:16.65067:9.172);
% \fill [black] (41.52,-24.14) -- (41.77,-23.23) -- (40.81,-23.51);
% \draw (39.74,-18.8) node [right] {$a/a$};
% \draw [black] (29.422,-35.081) arc (-103.86774:-256.13226:8.89);
% \fill [black] (29.42,-17.82) -- (28.53,-17.53) -- (28.77,-18.5);
% \draw (22.16,-26.45) node [left] {$a/a$};
% \end{tikzpicture}
% \end{center}

\section*{Decidable Word Problem}

\begin{definition} 
The word problem for finitely generated semigroups is as follows: given a f.g. semigroup $S$ with generators $\Sigma = s_1 \ldots s_n$, and two words $w_1, w_2$ over $\Sigma^*$, does $w_1 = w_2$ in $S$?
\end{definition} 
% The word problem is known to be undecidable for surprisingly small subclasses of (semi)groups. However, the word problem is decidable for the class of automaton semigroups.

% \begin{definition}
% % Acceptor DFA definition
% Given a transducer $A$ with some state $t$, the \textbf{acceptor of $A$ at $t$}, denoted $A(t)$ is a DFA constructed as follows: for every transition $a/b$ in $A$, $A(t)$ reads in the character $(a, b) \in \textbf{2}^*$. $t$ is $A(t)$'s start state, and state is an accept state.
% \end{definition}

% \begin{definition}
% For two transducers $A_1$, $A_2$, the product transducer $A_1\times A_2$ is the automaton $A$ with states $(s_1, s_2)$ for $s_1$ and $s_2$ states in $A_1$, $A_2$, respectively. The transition function is
% \[ \delta_a(\underline{(s_1, s_2)}) = (\delta_a \underline{s_1}, \delta_{a \underline{s_1}} \underline{s_2}) \]
% with $(s_1, s_2)$ a toggle state iff exactly one of $s_1$, $s_2$ is a toggle state.
% \end{definition}

\begin{proposition}
Automaton semigroups has a decidable word problem.
\end{proposition}
\begin{proofsketch}
We're given as input $w_1$, $w_2$.

The class of automaton semigroups is closed under direct products, so construct an automata whose state set denotes all words of length $|w_1|$, $|w_2|$ over the generators. Then choose states $w_1$ and $w_2$ in these automata as start states, and turn the transducers into DFAs. Minimize and check for equality.
\end{proofsketch}

\section*{Primary Results}

Most of our current work focuses on the \textsc{IsGroup} decision problem, whether a given automaton semigroup is already a group. Our work toward this goal is summarized below.

\section*{\textsc{IsGroup} is decidable in the Abelian case}

\begin{proofsketch} 
We can represent elements of the semigroup as vectors over $\mathbb{N}^{|Q|}$. Residuation becomes an affine operator, reducing the problem to matrix algebra.
\end{proofsketch} 

\section*{\textsc{Knapsack} is undecidable}

\begin{definition}
The \textsc{Knapsack} problem for automaton semigroups is as follows: given generators $s_1\ldots s_n \in S(A)$ and some element $s \in S(A)$, do there exist $a_1\ldots a_n \in \N$ such that \[s_1^{a_1}\ldots s_n^{a_n} = s \]
\end{definition}

\begin{theorem}
\textsc{Knapsack} is undecidable.
\end{theorem}

\begin{proofsketch}
Reduce from Hilbert's 10 problem.
\end{proofsketch}

\section*{A Monoid with undecidable \textsc{IsGroup}}

% This monoid is not an automaton monoid, but it has a decidable word problem, so it's at least related.
While the monoid is not an automaton monoid, it has a decidable word problem.

\begin{proofsketch}
Reduce from the Halting problem. Given an input TM $T$, define a group whose generators are configurations of $T$. Set the halting state to the identity. If one configuration transitions to another, their corresponding group elements are equal.

Then consider the submonoid generated by the start state $\langle s \rangle$. $\langle s \rangle$ is a group iff $T$ halts.

In order to keep the word problem decidable, we need to modify $T$ to be a self-verifying Turing machine.
% There's a bit more work to be done here, 
\end{proofsketch}

\section*{Open Questions}

\begin{itemize}
\item Is \textsc{IsGroup} decidable in the non-abelian case?
\item Is it decidable, given two automaton, whether they generate isomorphic semigroups? 
\item Automorphism membership problem: Given an endomorphism $\phi \in End(\textbf{2}^*)$ and an automata $A$, is $\phi \in S(A)$? Is this problem decidable?
\item A wide variety of other related semigroup decision problems.
\end{itemize}

\section*{Acknowledgements}

Thanks to Klaus Sutner for his guidance.

%--------------------------------------------------------------------------------------

\end{multicols}
\end{document}
