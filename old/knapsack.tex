\documentclass{article}
% \usepackage[margin=1in]{geometry}
\usepackage{amsmath, amssymb}
\usepackage{parskip}
\usepackage{tikz}
\usepackage{bussproofs}
\usepackage{enumerate}
\usepackage{inconsolata}
\usepackage[final]{microtype}
\usepackage{algorithm}
\usepackage[noend]{algpseudocode}

%%%%%%%%% Useful shorthand %%%%%%%%%%%%
\newcommand{\concourse}{}
\newcommand{\0}{\underline{0}}
\newcommand{\1}{\underline{1}}
\newcommand{\2}{\underline{2}}
\newcommand{\N}{\mathbb{N}}
\newcommand{\Z}{\mathbb{Z}}
\renewcommand{\S}{\mathcal{S}}
\renewcommand{\min}{\text{\concourse{min}}}
\renewcommand{\max}{\text{\concourse{max}}}
\renewcommand{\slash}{\text{ }/\text{ }}
\newcommand{\E}{\mathbb{E}}
\newcommand{\pr}[1]{\text{\concourse{Pr}}\left(#1\right)}
\newcommand{\Bernoulli}[1]{\text{\concourse{Bernoulli}}\left(#1\right)}
\newcommand{\geo}[1]{\text{\concourse{Geometric}}\left(#1\right)}
\newcommand{\expon}[1]{\text{\concourse{Exp}}\left(#1\right)}
\newcommand{\normal}[1]{\text{\concourse{Normal}}\left(#1\right)}
\newcommand{\uni}[1]{\text{\concourse{Uniform}}\left(#1\right)}
\newcommand{\var}[1]{\text{\concourse{Var}}\left(#1\right)}
\renewcommand{\exp}[1]{\mathbb{E}\left[#1\right]}
%%%%%%%%%%%%%%%%%%%%%%%%%%%%%%%%%%%%%%%

\title{The Knapsack Problem for Automatic Semigroups is Undecidable}
\author{Klaus Sutner \& Evan Bergeron}
\begin{document}
\maketitle
We reduce from Hilbert's tenth problem, over natural numbers. Fix a polynomial $P(x_1, \ldots x_n)$ such that the question ``is there a solution to $P(x_1, \ldots x_n) = a$'' is undecidable.

% \begin{algorithm}
% \begin{algorithmic}[1]
% \Procedure{HilbertsTenth}{}
% \EndProcedure
% \end{algorithmic}
% \end{algorithm}

% \begin{verbatim}
% def hilbertsTenth(a):
%   # P fixed
%   P_plus = all positive terms of P
%   P_neg = all negative terms of P
% \end{verbatim}

Take $P$ and separate it into $P_+$ and $P_-$, the positive and negative parts of $P$. Consider the equation $P_+ = P_-$.

\end{document}
