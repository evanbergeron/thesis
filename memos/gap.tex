\documentclass{article}
% \usepackage[margin=1in]{geometry}
\usepackage{amsmath, amssymb}
\usepackage{parskip}
\usepackage{tikz}
\usepackage{bussproofs}
\usepackage{enumerate}
\usepackage{inconsolata}
\usepackage[final]{microtype}

%%%%%%%%% Useful shorthand %%%%%%%%%%%%
\newcommand{\concourse}{}
\newcommand{\0}{\underline{0}}
\newcommand{\1}{\underline{1}}
\newcommand{\2}{\underline{2}}
\newcommand{\N}{\mathbb{N}}
\newcommand{\Z}{\mathbb{Z}}
\renewcommand{\S}{\mathcal{S}}
\renewcommand{\min}{\text{\concourse{min}}}
\renewcommand{\max}{\text{\concourse{max}}}
\renewcommand{\slash}{\text{ }/\text{ }}
\newcommand{\E}{\mathbb{E}}
\newcommand{\pr}[1]{\text{\concourse{Pr}}\left(#1\right)}
\newcommand{\Bernoulli}[1]{\text{\concourse{Bernoulli}}\left(#1\right)}
\newcommand{\geo}[1]{\text{\concourse{Geometric}}\left(#1\right)}
\newcommand{\expon}[1]{\text{\concourse{Exp}}\left(#1\right)}
\newcommand{\normal}[1]{\text{\concourse{Normal}}\left(#1\right)}
\newcommand{\uni}[1]{\text{\concourse{Uniform}}\left(#1\right)}
\newcommand{\var}[1]{\text{\concourse{Var}}\left(#1\right)}
\renewcommand{\exp}[1]{\mathbb{E}\left[#1\right]}
%%%%%%%%%%%%%%%%%%%%%%%%%%%%%%%%%%%%%%%

\title{Toward a proof of the gap theorem}
\author{Evan Bergeron \& Tim Becker}
\begin{document}
\maketitle
We found a couple of issues in the current proof sketch.

Does the parity of $p_w(c_1 + d)$ indicate whether $\mathfrak{A}(A, c_1 + d)$ copies the bit after $w$, starting at $2(c_1 + d)$? This seems critical to the proof's function, and does not seem to not be the case with the current definition. Changing our definition of $p_w$ to 
\[
p_{wa} = \begin{cases}
A \cdot p_w(x)&\text{if $p_w(x)$ is even}\\
A \cdot p_w(x)+(-1)^ax&\text{otherwise}\\
\end{cases}
\]
seems to ensure this (change the casing from $p_w(c_1)$ to $p_w(x)$).

$p_w$ is not a map from $\Z^m$ to $\Z^m$, as the input vectors could have one-halves in them (in particular $c_1$ is not integral). Instead, we define $p_\epsilon(x) = x$ and consider $p_w(2(c_1))$ and $p_w(2(c_1 + d))$.

Some notational confusion: What is $\mathfrak{C}(A, r)$? Is $\mathfrak{A}(A) = \mathfrak{A}(A, c_1)$?
\end{document}
